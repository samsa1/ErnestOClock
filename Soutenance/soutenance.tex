% !TeX spellcheck = fr-FR

\documentclass[intlimits, 10pt]{beamer}
\usetheme{Hannover}
\usecolortheme{crane}


%%% Pretty pretty
\setbeamercolor{alerted text}{fg=violet!80!black}
\setbeamercolor{background canvas}{bg=white}
\setbeamercolor{title}{fg=violet!80!black, bg=yellow!80!white}
\setbeamercolor{titlelike}{fg=violet, bg=yellow!80!white}
\setbeamercolor{sidebar}{bg=yellow!60!white}
\setbeamercolor{structure}{fg=violet!80!black}
\setbeamercolor{normal text}{fg=black}
\setbeamercolor{subsection in sidebar}{fg=violet!80!black}
\setbeamercolor{subsection in sidebar shaded}{fg=violet!40!gray}
\setbeamercolor{section in sidebar}{fg=violet!80!black}
\setbeamercolor{section in sidebar shaded}{fg=violet!40!gray}
\setbeamercolor{title in sidebar}{fg=violet!80!black}
\setbeamercolor{author in sidebar}{fg=violet!50!gray}
\setbeamercolor{block title}{fg=violet!80!black, bg=yellow!80!white}
\setbeamercolor{block body}{bg=yellow!15!white}

\addtobeamertemplate{navigation symbols}{}{%
	\usebeamerfont{footline}%
	\usebeamercolor[fg]{footline}%
	\hspace{1em}%
	\insertframenumber/\inserttotalframenumber
}

\usepackage{pfmathbeamer}
\usepackage{graphicx}
\usepackage{listings}
\lstset{
	backgroundcolor=\color{yellow!5!white},
	keywordstyle=\color{violet!80!black},
	numbers=left,
	frame=single,
	numberstyle=\tiny\color{violet!80!black},
	escapeinside={\%*}{*)},
	stringstyle=\color{black!10!yellow}
}


\hypersetup{pdfstartview=Fit}

\title[{\includegraphics[width=1.3cm, height=1.3cm]{../Rapport/logo} \\ \textsc{Ernest'O'Clock}}]{Projet de Systèmes Numériques}
\date{}
\author[\textsc{Fournier}, \textsc{Schabanel}, \textsc{Vivien}]{Paul \textsc{Fournier}, Juliette \textsc{Schabanel}, Samuel \textsc{Vivien}}

\DeclareMathOperator*{\argmin}{argmin}
\DeclareMathOperator*{\argmax}{argmax}
\newcommand{\bmm}[1]{\texorpdfstring{$\bm{#1}$}{#1}}

\AtBeginPart{\frame{\partpage}}
\AtBeginSection{\frame{\sectionpage}}
\AtBeginSubsection{\frame{\subsectionpage}}

\begin{document}
	\selectlanguage{french}
	
	\maketitle	
	
	\begin{frame}
		\tableofcontents
	\end{frame}
	
		\section{Compilation \texttt{.net} $\rightarrow$ \texttt{.c}}
	
	\begin{frame}{Compilation}
			\begin{table}
				\begin{tabular}{|c|c|}
				\hline
				Opérations sur les fils & Opérations sur les tableaux \\
				\hline
				Constante (\texttt{true} ou \texttt{false})& Constante ($0$ ou $1$)  \\
				\hline
				\texttt{not a}& \texttt{!a}  (nb: plus rapide qu'un \texttt{xor})\\
				\hline
				\texttt{and} & \texttt{\&}  \\
				\hline
				\texttt{or}& \texttt{|} \\
				\hline
				\texttt{xor}& \texttt{\^} \\
				\hline
				\texttt{a nand b }& \texttt{!(a \& b)} \\
				\hline
				\texttt{mux s b a}& \texttt{s?a:b} \\
				\hline
				\texttt{reg}&Assignation\\
				\hline
			\end{tabular}
			\caption{Traduction \texttt{.net} $\rightarrow$ \texttt{.c}}
		\end{table}
	\end{frame}


	\section{Microprocesseur}
	
	\begin{frame}{Caractéristiques}
		\begin{block}{Grandes idées}
			\begin{itemize}
				\item Peu d'instructions (pas de multiplication notamment)
				\item 32 registres de 32 bits
				\item Les mêmes drapeaux qu'en x86-64
				\item Code dans la ROM, mémoire dans la RAM
			\end{itemize}
		\end{block}
	\end{frame}
	
	\begin{frame}{Code machine}
		\begin{block}{Format des instructions}
			\begin{itemize}	
				\item 7 bits de code machine
				\item 1 bit d'écriture
				\item 16 bits de valeur d'entrée
				\item 2 bits de type de valeur d'entrée
				\item 5 bits de valeur de sortie
				\item 1 bit de valeur de sortie
			\end{itemize}
		\end{block}
	\end{frame}
	
	\section{Langage assembleur}
	
	\begin{frame}{Opérations arithmétiques et booléennes}
		\begin{itemize}
		\item \texttt{add} $x \text{ } y \rightarrow x := x + y$
		\item \texttt{sub} $x \text{ } y \rightarrow x := x - y$
		\item \texttt{neg} $x \rightarrow x := - x $
		\item \texttt{and} $x \text{ } y \rightarrow x := x \&\& y$
		\item \texttt{or} $x \text{ } y \rightarrow x := x || y$
		\item \texttt{xor} $x \text{ } y \rightarrow x := x ^ y$
		\item \texttt{not} $x \rightarrow x := \neg x$
		\item \texttt{lsl / lsr} $x \text{ } n \rightarrow $ logical shift de $n$
		\item \texttt{asr} $x \text{ } n \rightarrow $ arithmetical shift de $n$
		\item \texttt{incr} $x \rightarrow x := x + 1$
		\item \texttt{decr} $x \rightarrow x := x - 1$
		\end{itemize}
	\end{frame}
	
	\begin{frame}{Autres opérateurs}
		\begin{itemize}
		\item \texttt{mov[flag]} $x \text{ } y \rightarrow x := y$ conditionnée par le drapeau
		\item \texttt{jmp / j[flag]} $label \rightarrow $ saut (conditionnel) vers le label \\
		\item \texttt{cmp} $x \text{ } y \rightarrow $ donne aux drapeaux les valeurs pour $x - y$
		\item \texttt{test} $x \text{ } y \rightarrow $ donne aux drapeaux les valeurs pour $x \&\& y$
		\end{itemize}
	\end{frame}

		\begin{frame}{Les drapeaux}
	\begin{itemize}
		\item \texttt{e} : nullité du résultat [ZF]
		\item \texttt{ne} : non nullité [$\neg$ ZF]
		\item \texttt{s} : résultat négatif [SF]
		\item \texttt{ns} : résultat positif [$\neg$ SF]
		\item \texttt{g} : > [$\neg$(SF xor OF) \& $\neg$ ZF] 
		\item \texttt{ge} : $\geqslant$ [$\neg$(SF xor OF)]
		\item \texttt{l} : < [SF xor OF]
		\item \texttt{le} : $\leqslant$ [(SF xor OF)|ZF]
		\item \texttt{a} : > non signé [$\neg$ CF \& $\neg$ ZF]
		\item \texttt{ae} : $\geqslant$ non signé [$\neg$ CF]
		\item \texttt{b} : < non signé[CF]
		\item \texttt{be} : $\leqslant$ non signé [CF$\vert$ZF]
	\end{itemize}
	\end{frame}
		
	\begin{frame}{Syntaxe}
		\begin{itemize}
		\item[$\bullet$] $\%<nom \text{ } du \text{ } registre>$ pour désigner la valeur contenue dans un registre.
		\item[$\bullet$] $(\%<nom \text{ } du \text{ } registre>)$ pour désigner la valeur contenue en mémoire à  l'adresse contenue dans le registre, un seul de ces accès mémoire est autorisé par instruction
		\item[$\bullet$] $\$<entier>$ pour les constantes.
		\item[$\bullet$] $<label>:$ pour poser un label et $"<label>"$ pour y sauter, les caractère autorisés pour les noms de labels sont les lettres majuscule et minuscules de l'alphabet latin, '\_' et les chiffres hors première caractère (i.e. \texttt{(['a'-'z' 'A'-'Z'] $\vert$ '\_') (['a'-'z' 'A'-'Z'] $\vert$ '\_' $\vert$ ['0'-'9'])*}).
		\item[$\bullet$] $\#$ pour désigner le début d'un commentaire, tout le texte suivant un $\#$ sera ignoré jusqu'au premier retour à  la ligne.
	\end{itemize}
	\end{frame}
	
	\begin{frame}{Grammaire}
		La grammaire est décrite simplement par les règles suivantes :
	$$\begin{array}{rcl}
		<mem> & = & \%<register> | (\%<register>) \\
		<param> & = &  \%<register> | (\%<register>) | \$<entier>\\
		<instr> & = & <operateur> <mem> <param>? \\
		& | & mov <flag>? <mem> <param> \\
		& | & jmp <flag>? "<label>" \\
		& | & <label> : \\ 
	\end{array}$$
	\end{frame}
	
	\begin{frame}{Assembleur}
		Réalisé en \texttt{OCaml} avec \texttt{ocamllex} et \texttt{menhir}.\\
		Vérifie la limite d'un accès mémoire par instruction.
	\end{frame}

	
	\section{Interface homme machine}
	
	\begin{frame}{Format de l'affichage}
		\begin{figure}
			\centering
			\includegraphics[width=0.7\linewidth]{../Rapport/2021-01-18_16-19}
			\caption{Démonstration statique de l'horloge}
			\label{fig:2021-01-1816-19}
		\end{figure}
		
	\end{frame}
	
	\begin{frame}{Memory-Mapped-Input-Output}
		\begin{table}
			\begin{tabular}{|c|c|c|c|c|c|}
				\hline
				$0-6$ & $7-13$ & $14-20$ & $21-27$ & $30$ & $31$ \\ 
				\hline
				Sec. 1 & Sec. 10 & Min. 1 & Min. 10 & Affichage? & Parité \\
				\hline
				\hline
				$33-39$ &$41-47$&$49-55$ &$57-63$ && \\
				\hline
				Heure 1 & Heure 10 & Jour 1 & Jour 10 &&\\
				\hline
				\hline
				$65-71$ & $73-79$ & $81-87$ & $89-95$ &&\\
				\hline
				Mois 1 & Mois 10 & Année 1 & Année 10 &&\\
				\hline
				\hline
				$128-159$ &&&&&\\
				\hline
				Init &&&&& \\
				\hline
			\end{tabular}
			\caption{Description de l'interface mémoire I/O}
		\end{table}
	\end{frame}
	
	\begin{frame}{Format MMIO}
		
		\begin{figure}
			\centering
			\includegraphics[width=0.3\linewidth]{../Rapport/segments}
			
			\begin{tabular}{|c|c|c|c|c|c|c|c|}
				\hline
				$b_0$& $b_1$  & $b_2$ & $b_3$ & $b_4$ & $b_5$ & $b_6$ & $b_7$ \\
				\hline
			\end{tabular}
			\caption{Représentation binaire de la sortie}
			\label{fig:segments}
		\end{figure}
		
		

	\end{frame}
	
	\section{Horloge}

		\begin{frame}[fragile, shrink]{Initialisation de l'horloge}
		\begin{lstlisting}[language={[x86masm]Assembler},morekeywords={lsr}, caption="Division euclidienne"]
			mov %r26 $86400
			mov %r25 %r26
			jmp "ew1"
			bw1:
			lsl %r25 $1
			ew1:
			cmp %r28 %r25
			jb "bw1"
			mov %r27 $0
			bw2:
			lsl %r27 $1
			cmp %r28 %r25
			ja "ew2"
			add %r27 $1
			sub %r28 %r25
			ew2:
			lsr %r25 $1
			cmp %r25 %r26
			jge "bw2"
		\end{lstlisting}
	\end{frame}
	

	\begin{frame}{Incrémentation}
	
	\begin{figure}[h]
		\centering
		\includegraphics[width=0.9\linewidth]{Schema_incrementation}
		\caption{Schéma d'incrémentation}
	\end{figure}

	\end{frame}
		
	\begin{frame}{Schéma de test pour les mois}
			$$
	\begin{tikzpicture}[baseline = -20pt, level distance = 0.75cm, level 1/.style={sibling distance=3cm},
                   level 2/.style={sibling distance=2cm},
                   level 3/.style={sibling distance=1.5cm},
                   level 4/.style={sibling distance=0.75cm}]
	\begin{scope}[every node/.style={fill=white, inner sep=1pt}]
	\node {$\geqslant 8\text{ ?}$}
	 child {node {$\text{pair ?}$}
	        child {node{$31$}}
	        child {node{$= 2\text{ ?}$}
	        		child {node{$30$}}
			child {node{\text{bissextile ?}}
				child {node{$28$}}
				child {node{$29$}}
				}
			}
	       }
	 child {node {$\text{pair ?}$}
	 	child {node{$30$}}
		child {node{$31$}}
		}
	      ;
	\end{scope}
	\end{tikzpicture}
	$$
	\end{frame}
	
	\begin{frame}{Output}
		$$
	\begin{tikzpicture}[baseline = -20pt, level distance = 0.75cm, level 1/.style={sibling distance=3cm},
                   level 2/.style={sibling distance=2cm},
                   level 3/.style={sibling distance=1cm},
                   level 4/.style={sibling distance=0.5cm}]
	\begin{scope}[every node/.style={fill=white, inner sep=1pt}]
	\node {$s_3$}
	 child {node {$s_2$}
	        child {node{$s_1$}
	        		child {node{$s_0$}
				child {node{$0$}}
				child {node{$1$}}
				}
			child {node{$s_0$}
				child {node{$2$}}
				child {node{$3$}}
				}
			}
	        child {node{$s_1$}
	        		child {node{$s_0$}
				child {node{$4$}}
				child {node{$5$}}
				}
			child {node{$s_0$}
				child {node{$6$}}
				child {node{$7$}}
				}
			}
	       }
	 child {node {$s_0$}
	 	child {node{$8$}}
		child {node{$9$}}
		}
	      ;
	\end{scope}
	\end{tikzpicture}
	$$
	\end{frame}
	

\end{document}
